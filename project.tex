\documentclass[a4paper]{report}
\usepackage{graphicx}
\usepackage{booktabs}

\title{Foreign Exchange Management System (FXMS)}
\author{
    Yazeed AlKhalaf \\
    Mohammed Bin Jebreen \\
    Nawaf AlAmer \\
    \textbf{Course:} SWE 301 - Requirements Engineering \\
    \textbf{Instructor:} Dr. Noureddine Abbadeni
}
\date{20 Feb, 2024}

\begin{document}

\maketitle

\newpage

\tableofcontents

\chapter{System Request (FXMS)}

\section{Project Sponsor}
Dr. Noureddine Abbadeni

\section{Business Need}
The need for a project like the Foreign Exchange Management System (FXMS) is crucial for businesses operating internationally for several reasons:
\begin{itemize}
    \item Operating internationally: Businesses engaged in importing and exporting goods and services will need a system like FXMS for currency conversion, enabling them to exchange their local currency for that of the country in which they wish to operate, thereby settling international transactions.
    \item Managing cash flow: Businesses operating overseas need to manage their cash across multiple currencies. FXMS will help them monitor and optimize their cash by converting currency at favorable rates and timings.
    \item Softening the risk: FXMS will provide businesses with tools to manage and mitigate the risks associated with fluctuations in currency prices. By using specific strategies, companies can lower the risk of exchange rate volatility and protect their profit margins.
\end{itemize}

\section{Business Requirements}
The functionality that the system should have includes:
\begin{itemize}
    \item Ability to manage clients and accounts (insert, update, delete).
    \item Ability to manage trades (insert, update, and delete trades). Any trader can enter new trades while updating and deleting existing trades require specific privileges.
    \item Ability to manage traders and coverage groups by assigning a trader to a coverage group, moving a trader from one coverage group to another.
    \item Ability to manage currencies and rates including daily updates of rates available in the market. The system is assumed to be connected with another system (such as Tadawul) which provides daily updates for exchange rates between all currencies.
\end{itemize}

\section{Business Value}
The Foreign Exchange Management System (FXMS) is expected to deliver some gains:
\begin{itemize}
    \item Quicker and Better Decision Making: Taking good decisions in a quick manner gives a competitive advantage in the international markets.
    \item Less Human Error: The human factor will be limited to things that require humans interaction and not things that are repetitive that are error prone.
    \item More Money: The amount of money traded will be more giving the organization a better chance at making more money.
    \item Headcount reduction by 10 traders per branch.
    \item 15\% increase in market share.
\end{itemize}

\section{Constraints}
\begin{itemize}
    \item The system should run on Windows 10.
    \item The system should be delivered by the end of the year 2028.
    \item Security and reliability must be considered during development.
\end{itemize}

\chapter{Feasibility Study}

Overall, the risk in this project compared to the gains can be considered manageable.

\section{Technical}

The technical team is confident they can build it since they built a similar system before, the knowldege they gained during that experience lowers the risk.

\begin{itemize}
    \item Familiarity with application: The team is familiar with building an FXMS.
    \item Familiarity with technology: Since the team members have a collective experience of over 50 years building complex software, we are confident they will be able to tackle the project.
    \item Project Size: Large project, but since team is familiar, it won't be a high risk as usual.
    \item Compatibility: The company wants a custom solution, so we will make sure it integrates well by analysing before we build anything and before we choose a platform.
\end{itemize}

\section{Financial}

We will start with the big numbers, the ROI and the BEP.

\begin{figure}[h!]
    \centering
    \includegraphics[width=0.8\textwidth]{images/roi-bep.png}
    \caption{ROI and BEP of FXMS}
    \label{fig:roi-and-bep}
\end{figure}

\subsection{Cost-Benefit Analysis}

The cashflow analysis below in Figure \ref{fig:cash-flow-analysis} is a condensed versin of the 4 years (monthly based) version of the cashflow analysis. It gives an idea on the way the project will behave financially.

\begin{figure}[h!]
    \centering
    \includegraphics[width=0.8\textwidth]{images/cash-flow-analysis.png}
    \caption{Cashflow Analysis of FXMS}
    \label{fig:cash-flow-analysis}
\end{figure}

\chapter{Methodology}

Below in Table \ref{tab:methodology-criteria}, the criteria we used to choose our methodology are mentioned with what we chose. 

\begin{table}[htbp]
    \centering
    \caption{Criteria Evaluation for System Development}
    \label{tab:methodology-criteria}
    \begin{tabular}{@{}p{0.7\linewidth}c@{}}
        \toprule
        Criteria & Answer \\
        \midrule
        Are the requirements clear? & Yes \\
        Is the technology familiar to the team? & Yes \\
        Is the system complex? & Yes \\
        Does the system need to be reliable? & Yes \\
        Is the system scheduled to be built in a short time? & No \\
        Do we have schedule visibility? & Yes \\
        \bottomrule
    \end{tabular}
\end{table}

We decided to go with V-Model methodology since it is simple and straightforward, and the testing phase ensures quality and reliability, in addition to the quality personnel and the engineers themselves who will bake the quality in. We don't beleive in doing quality work after the fact since it should be built and baked in from the beginning.

Also since the project requirements are clear and the team is comfortable with the technology, the V-Model methodology fits the use case and helps the project succeed.

\chapter{Project Workplan}

\end{document}
